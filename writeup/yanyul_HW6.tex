%Automation of Biological Research 
%HW6
%due Dec/9/2016
%Email responses to: ABR-instructors@googlegroups.com
%Landmark Clustering
\documentclass{article}
\usepackage{amsmath}
\usepackage{amssymb}
\usepackage{amsfonts}
\usepackage{graphicx}
\usepackage{listings}
\usepackage{qtree}
\usepackage{pgf}
\usepackage{tikz}
\usetikzlibrary{arrows,automata}
\usepackage[latin1]{inputenc}
\usepackage{verbatim}
\usepackage{fullpage}
\usepackage{algpseudocode}
\usepackage{algorithm}
\usepackage{cleveref}
\usepackage{courier}


%\usetikzlibrary{arrows}

\usepackage{indentfirst}
\newcommand{\E}{\mathrm{E}}
\newcommand{\Var}{\mathrm{Var}}
\newcommand{\Cov}{\mathrm{Cov}}
\newcommand{\T}{\mathrm{T}}
\newcommand{\F}{\mathrm{F}}
\newcommand\independent{\protect\mathpalette{\protect\independenT}{\perp}}
\def\independenT#1#2{\mathrel{\rlap{$#1#2$}\mkern2mu{#1#2}}}
%\setlength{\parindent}{1cm} % Default is 15pt.



\begin{document}

\title{Automation of Biological Research HW6-2016fall}
\author{Yanyu Liang\quad AndrewID: yanyul}
\maketitle


\section*{Question 1 - Passive Clustering - Synthetic Data}

\subsection*{Task C}

  The plots are shown in \Cref{fig:q1a} and \Cref{fig:q1b}.

  \begin{figure}[!ht]
    \begin{minipage}{.45\textwidth}
      \centering
      \includegraphics[width=1\linewidth]{q1_keams}
      \caption{Plain Kmeans}
      \label{fig:q1a}
    \end{minipage}
    \hfill
    \begin{minipage}{.45\textwidth}
      \centering
      \includegraphics[width=1\linewidth]{q1_spectral}
      \caption{Spectral Clustering Approach}
      \label{fig:q1b}
    \end{minipage}
  \end{figure}

  The decision boundary of the Kmeans approach is a straight line (because we are using Euclidean distance with two clusters) but the decision boundary of the spectral clustering approach is definitely not a straight line. The difference comes from the fact that spectral clustering maps the points in original space to another space with nonlinear mapping and in that space, the distance boundary is linear, which results in a nonlinear distance boundary in original space.

\subsection*{Task D}

  The number of iterations in the Kmeans approach is 10, then the number of pairwise distance calculations is $10 \times 3384^2 = 114514560$. For the spectral clustering approach, the preprocessing (mapping) needs $n^2$ times pairwise calculations and the number of Kmeans iterations is 2, then the overall number of pairwise distance calculations is $3 \times 3384^2 = 34354368$.

\section*{Question 2}

\subsection*{Task C}

  The plots are shown in \Cref{fig:q2a}, \Cref{fig:q2b}, \Cref{fig:q2c}, and \Cref{fig:q2d}.

  \begin{figure}[!ht]
    \begin{minipage}{.45\textwidth}
      \centering
      \includegraphics[width=1\linewidth]{q2_keams}
      \caption{Plain Kmeans}
      \label{fig:q2a}
    \end{minipage}
    \hfill
    \begin{minipage}{.45\textwidth}
      \centering
      \includegraphics[width=1\linewidth]{q2_spectral}
      \caption{Spectral Clustering Approach}
      \label{fig:q2b}
    \end{minipage}
    \vfill
    \begin{minipage}{.45\textwidth}
      \centering
      \includegraphics[width=1\linewidth]{q2_landmark}
      \caption{Landmark Clustering Approach}
      \label{fig:q2c}
    \end{minipage}
    \hfill
    \begin{minipage}{.45\textwidth}
      \centering
      \includegraphics[width=1\linewidth]{q2_truth}
      \caption{Ground Truth}
      \label{fig:q2d}
    \end{minipage}
  \end{figure}

  The decision surface of the landmark clustering approach resembles the plain Kmeans approach most. The general procedure of the landmark clustering approach is i) to find cluster centers with reduced number of pairwise distances; ii) assign each point to the clusters. So, the way it assigns points is the same as the one in the Kmeans approach, therefore their decision surface resemble to each other.

\subsection*{Task D}

  The number of pairwise distance queries made by the landmark clustering approach is 379008.

\section*{Question 3}

\subsection*{Task A}

  The plots are shown in \Cref{fig:q3a}, \Cref{fig:q3b}, \Cref{fig:q3c}, and \Cref{fig:q3d}. And the plain Kmeans approach seems to work best.

  \begin{figure}[!ht]
    \begin{minipage}{.45\textwidth}
      \centering
      \includegraphics[width=1\linewidth]{q3_keams}
      \caption{Plain Kmeans}
      \label{fig:q3a}
    \end{minipage}
    \hfill
    \begin{minipage}{.45\textwidth}
      \centering
      \includegraphics[width=1\linewidth]{q3_spectral}
      \caption{Spectral Clustering Approach}
      \label{fig:q3b}
    \end{minipage}
    \vfill
    \begin{minipage}{.45\textwidth}
      \centering
      \includegraphics[width=1\linewidth]{q3_landmark}
      \caption{Landmark Clustering Approach}
      \label{fig:q3c}
    \end{minipage}
    \hfill
    \begin{minipage}{.45\textwidth}
      \centering
      \includegraphics[width=1\linewidth]{q3_truth}
      \caption{Ground Truth}
      \label{fig:q3d}
    \end{minipage}
  \end{figure}

\subsection*{Task B}

  The $(c, \epsilon)$-property is the property that for a given data set, if the clustering (assignments) is within $c$-suboptimal, then the fraction of wrongly assigned points is bounded by $\epsilon$. In other word, it requires that the underlying clustering of the data should be almost separable, which means that there should not be so many points near the decision surface. The property depends on both the data and the metric. For biological datasets, $(c, \epsilon)$-property is not always holds. For instance, many proteins family may share certain common domain, and this leads to the overlap of the clusters and makes the property hardly hold. 

  In our data set, from the SVM result we can see, cluster 4 and 5 is not separable under $l_2$ metric and roughly over 80\% of the points are near boundary in cluster 4 and 5. Therefore, it fails to follow $(c, \epsilon)$-property.

\subsection*{Task C}

  If $s_{\min}$ is too small, then the outliers can also form balls during the landmark expansion, which may lead to the method wrongly merge two clusters. If $s_{\min}$ is too big, the landmark expansion may fail and it can also lead to some unexpected merging.

\end{document}